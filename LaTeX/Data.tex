\subsection{Generating Input Data}
Next we need to prepare data to test our program. 
A function will be used to generate a matrix of sample inputs based on the desired sample size and the feature size. 
Feature size is the count of independent elements.
\begin{lstlisting}
    def generate_in(sample_size, feature_size):
        data=[[0]*feature_size for i in range(sample_size)]
        for i in range(sample_size):
            for j in range(feature_size):
                data[i][j]=np.random.randint(100)
        X=Matrix(data)
        return X
\end{lstlisting}
\subsection{Generating Output Data}
Another function will generate the output for the respective inputs genertaed above.
The function will be passed the sample input as well as a coeffecient matrix. 
Then noise will be added to the function to simulate real world conditions.

\begin{lstlisting}
    def generate_out(A,X):
        Y=X*A
        add_noise(Y,5)
        return Y

    def add_noise(A,max):
        for i in range(A.m):
            for j in range(A.n):
                A.matrix[i][j]=A.matrix[i][j]+np.random.rand()*max
\end{lstlisting}